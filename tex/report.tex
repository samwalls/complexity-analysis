\documentclass[]{article}
\usepackage{times}
\usepackage[margin=1in]{geometry}
\usepackage{algorithm2e}
\usepackage{dsfont}

%opening
\title{CS3052 - Computational Complexity -\\Analysis of the Discrete Fourier Transform}
\author{ID:150013828}

\begin{document}

\maketitle

\section{Overview}
In this report, we give a look to each provided algorithm in turn. First, an \emph{a priori} analysis is provided for the worst-case time complexity of each - and subsequently, an empirical analysis of the average case of time complexity is provided.
\\\\
Afterwards, an observation of the threshold for which FFT outperforms naive DFT is described in section \ref{sec:threshold}.
\section{DFT Analysis}\label{sec:dft}
\subsection{Worst Case Time Complexity}
First, the algorithm in question is shown in pseudocode in figure \ref{alg:dft}.
\begin{algorithm}[h]
	\KwIn{$x_0, x_1, ..., x_{N-1} : x \in \mathds{C}$}
	$X = \verb|the empty list|$\\
	\For{$i = 0$; $i < N$}{
		$X_i = 0$\\
		\For{$j = 0$; $j < N$}{
			$z = x_n e^{-2kn\pi i / N} = x_n(\cos(\frac{-2kn\pi}{N}) + i \sin(\frac{-2kn\pi}{N}))$\\
			$X_i \leftarrow X_i + z$\\
			$j \leftarrow j + 1$\\
		}
		$i \leftarrow i + 1$\\
	}
	\KwResult{X}
\caption{The naive DFT algorithm\label{alg:dft}}
\end{algorithm}

For this algorithm, the worst-case time complexity is quite clear. As can be seen in the pseudocode, the algorithm contains one \emph{for-loop} nested inside another - both of which iterate $N$ times, where $N$ is the size of the input list. Hence, the total number of evaluations of the inner loop is $N^2$.

Knowing that the code within the loop does not break at an early point - we can say that the time taken for an input list of size $n$, $T(n)$, is at the least, proportional to $n^2$. Now, with these facts in mind, it is reasonable to assume that the algorithm is of order $O(n^2)$ in worst-case time complexity.

\subsection{Average Case Analysis}
\section{FFT Analysis}\label{sec:fft}
\subsection{Worst Case Time Complexity}
\subsection{Average Case Analysis}
\section{Empirical Observation for the DFT/FFT Threshold}\label{sec:threshold}
\begin{figure}[h]
	\begin{center}
		\begin{tabular}{ | c | c | }
			\hline
			specification & value \\
			\hline
			CPU & x86\_64 Intel(R) Core(TM) i7-4712HQ CPU (4 physical cores) @ 2.30GHz \\
			\hline
			RAM & 16GB = 2 $\times$ 8GB @ 1600MHz DDR3\\ 
			\hline
			OS & Ubuntu \emph{xenial} 16.04.2 \\
			\hline
			Java Version & 1.8.0\_121 \\
			\hline
		\end{tabular}
	\end{center}
	\caption{relevant system specifications used in provided analysis\label{fig:specs}}
\end{figure}


\end{document}
