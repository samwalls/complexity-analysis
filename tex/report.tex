\documentclass[]{article}
\usepackage{times}
\usepackage[margin=1in]{geometry}
\usepackage{algorithm2e}
\usepackage{dsfont}

%opening
\title{CS3052 - Computational Complexity -\\Analysis of the Discrete Fourier Transform}
\author{ID:150013828}

\begin{document}

\maketitle

\section{Overview}
In this report, we give a look to each provided algorithm in turn. First, an \emph{a priori} analysis is provided for the worst-case time complexity of each - and subsequently, an empirical analysis of the average case of time complexity is provided.
\\\\
Afterwards, an observation of the threshold for which FFT outperforms naive DFT is described in section \ref{sec:threshold}.
\section{DFT Analysis}\label{sec:dft}
\subsection{Worst Case Time Complexity}
First, the algorithm in question is shown in pseudocode in figure \ref{alg:dft}.
\begin{algorithm}[h]
	\KwIn{$x_0, x_1, ..., x_{N-1} : x \in \mathds{C}$}
	$Y = \verb|the empty list|$\\
	\For{$j = 0$; $j < N$}{
		$Y_j = 0$\\
		\For{$k = 0$; $k < N$}{
			$z = x_k e^{-2kn\pi i / N} = x_n(\cos(\frac{-2kn\pi}{N}) + i \sin(\frac{-2kn\pi}{N}))$\\
			$Y_j \leftarrow Y_j + z$\\
			$k \leftarrow k + 1$\\
		}
		$j \leftarrow j + 1$\\
	}
	\Return{X}
\caption{The naive DFT algorithm\label{alg:dft}}
\end{algorithm}

For this algorithm, the worst-case time complexity is quite clear. As can be seen in the pseudocode, the algorithm contains one \emph{for-loop} nested inside another - both of which iterate $N$ times, where $N$ is the size of the input list. Hence, the total number of evaluations of the inner loop is $N^2$.
	
Knowing that the code within the loop does not break at an early point - we can say that the time taken for an input list of size $n$, $T(n)$, is at the least, proportional to $n^2$. The evaluation of the inner loop will be of a constant factor, this is because the operations, $\sin$, $\cos$, addition, multiplication etc. will always take the same time within this context. I.e. we do not use arbitrary precision floating point numbers, the size is \emph{fixed}. So, it is reasonable to say that the inner loop is of a constant order complexity.

Now we know that: the inner loop will always run $N^2$ times; and that the evaluation of the inner loop is of the order $O(1)$ with respect to a single element of the input list. With all of these facts in mind, it is reasonable to assume that the naive DFT algorithm is of the order $O(n^2)$ in the worst case, where $n$ is the length of the input list. Not only this, but the average case, and even the best case, should be more or less the same as the worst case - this is because there is no room for pathological behaviour in the algorithm, it \emph{always} runs the inner loop $N^2$ times. That is to say, the algorithm is $\Omega(n^2)$, and $O(n^2)$, thus also $\Theta(n^2)$.

\subsection{Average Case Analysis}

\section{FFT Analysis}\label{sec:fft}

\subsection{Worst Case Time Complexity}
As before, the algorithm in question is shown in pseudocode in figure \ref{alg:fft}.
\begin{algorithm}[h]
	\KwIn{$x_0, x_1, ..., x_{N-1} : x \in \mathds{C}$}
	\If{N = 1}{
		\Return $x_0$
	}
	\If{}{
		\Return error: $N$ should be even
	}
	$q = \verb|FFT(all even terms in input list)|$\\
	$r = \verb|FFT(all odd terms in input list)|$\\
	Y
	\For{$j = 0$; $j < N/2$}{
		$z = r_n e^{-2kn\pi i / N} = x_n(\cos(\frac{-2kn\pi}{N}) + i \sin(\frac{-2kn\pi}{N}))$\\
		$Y_j \leftarrow Y_j + z$\\
		$Y_{j + N/2} \leftarrow Y_j - z$\\
		$j \leftarrow j + 1$\\
	}
	\Return Y
	\caption{The Fast Fourier Transform (FFT) algorithm
\label{alg:fft}}
\end{algorithm}

In order to analyse the complexity of the FFT algorithm, we will apply the \emph{master theorem} - as shown in the lectures of this module. The time taken for the FFT algorithm is proportional to the following recurrence relation:
$$T(1) = 1$$
$$T(n) = 2T(\frac{n}{2}) + \frac{n}{2}$$
where $T(n)$ is the time taken for an input list of size $n$.

\subsection{Average Case Analysis}

\section{Empirical Observation for the DFT/FFT Threshold}\label{sec:threshold}
\begin{figure}[h]
	\begin{center}
		\begin{tabular}{ | c | c | c | }
			\hline
			specification & PC1 Value & PC2 Value \\
			\hline
			CPU & x86\_64 Intel(R) i7-4712HQ CPU @ 2.30GHz & x86\_64 Intel(R) i5-3470S CPU @ 2.90GHz\\
			\hline
			RAM & 16GB = 2 $\times$ 8GB @ 1600MHz DDR3 & \\ 
			\hline
			OS & Ubuntu \emph{xenial} 16.04.2 & \\
			\hline
			Java Version & 1.8.0\_121 & 1.8.0\_121\\
			\hline
		\end{tabular}
	\end{center}
	\caption{relevant system specifications used in provided analysis\label{fig:specs}}
\end{figure}

\end{document}
