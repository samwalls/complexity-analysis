\documentclass[]{article}
\usepackage{times}
\usepackage[margin=1in]{geometry}
\usepackage{algorithm2e}
\usepackage{dsfont}
\usepackage{amsmath}
\usepackage{hyperref}
\usepackage{pgf}
\usepackage{tikz}
\usepackage{subfig}
\usepackage{pgfplots}
\usepackage{filecontents}
\usepackage[round, sort, numbers]{natbib}

%opening
\title{CS3052 - Computational Complexity -\\Analysis of the Discrete Fourier Transform}
\author{ID:150013828}

\begin{document}

\maketitle

\section{Overview}
In this report, we give a look to each provided algorithm in turn. First, an \emph{a priori} analysis is provided for the worst-case time complexity of each - and subsequently, an empirical analysis of the average case of time complexity is provided.
\\\\
Afterwards, an observation of the threshold for which FFT outperforms naive DFT is described in section \ref{sec:threshold}.

\section{DFT Analysis}\label{sec:dft}
\subsection{Worst Case Time Complexity}
First, the algorithm in question is shown in pseudocode in algorithm \ref{alg:dft}.
\begin{algorithm}[h]
	\KwIn{$x_0, x_1, ..., x_{N-1} : x \in \mathds{C}$}
	$Y = \verb|the empty list|$\\
	\For{$j = 0$; $j < N$}{
		$Y_j = 0$\\
		\For{$k = 0$; $k < N$}{
			$z = e^{-2jk\pi i / N} = \cos(\frac{-2jk\pi}{N}) + i \sin(\frac{-2jk\pi}{N})$\\
			$Y_j \leftarrow Y_j + zx_k$\\
			$k \leftarrow k + 1$\\
		}
		$j \leftarrow j + 1$\\
	}
	\Return{X}
\caption{The naive DFT algorithm\label{alg:dft}}
\end{algorithm}

For this algorithm, the worst-case time complexity is quite clear. As can be seen in the pseudocode, the algorithm contains one \emph{for-loop} nested inside another - both of which iterate $N$ times, where $N$ is the size of the input list. Hence, the total number of evaluations of the inner loop is $N^2$.
	
Knowing that the code within the loop does not break at an early point - we can say that the time taken for an input list of size $n$, $T(n)$, is at the least, proportional to $n^2$. The evaluation of the inner loop will be of a constant factor, this is because the operations, $\sin$, $\cos$, addition, multiplication etc. will always take the same time within this context. I.e. we do not use arbitrary precision floating point numbers, the size is \emph{fixed}. So, it is reasonable to say that the inner loop is of a constant order complexity.

Now we know that: the inner loop will always run $N^2$ times; and that the evaluation of the inner loop is of the order $O(1)$ with respect to a single element of the input list. With all of these facts in mind, it is reasonable to assume that the naive DFT algorithm is of the order $O(n^2)$, where $n$ is the length of the input list. Not only this, but the average case, and even the best case, should be asymptotically the same as the worst case - this is because there is no room for pathological behaviour in the algorithm, it \emph{always} runs the inner loop $N^2$ times. That is to say, the algorithm is $\Omega(n^2)$, and $O(n^2) \Rightarrow \Theta(n^2)$.

\subsection{Average Case Analysis}\label{sec:dft-average}

\begin{figure}[!htbp]
	\centering
	\begin{minipage}[b]{0.4\textwidth}
		\begin{tikzpicture}
		\begin{axis}[ xmin=0, ymin=0, xlabel=Input Length, ylabel=Time ($\mu$s), legend style={at={(1,-0.1)},anchor=north} ]
		\addplot table [ignore chars=\%, x=Param: inputLength, y=Score, col sep=comma] {csv/pc1-result-dft.csv};
		\addlegendentry{PC1}
		\end{axis}
		\end{tikzpicture}
		\caption{Naive DFT - PC1\label{fig:compar1}}
	\end{minipage}
	\hfill
	\begin{minipage}[b]{0.4\textwidth}
		\begin{tikzpicture}
		\begin{axis}[ xmin=0, ymin=0, xlabel=Input Length, ylabel=Time ($\mu$s), legend style={at={(1,-0.1)},anchor=north} ]
		\addplot table [ignore chars=\%, x=Param: inputLength, y=Score, col sep=comma] {csv/pc2-result-dft.csv};
		\addlegendentry{PC2}
		\end{axis}
		\end{tikzpicture}
		\caption{Naive DFT - PC2\label{fig:compar2}}
	\end{minipage}
\end{figure}

\section{FFT Analysis}\label{sec:fft}

\subsection{Worst Case Time Complexity}
As before, the algorithm in question is shown in pseudocode in algorithm \ref{alg:fft}.
\begin{algorithm}[h]
	\KwIn{$x_0, x_1, ..., x_{N-1} : x \in \mathds{C}$}
	\If{N = 1}{
		\Return $x_0$
	}
	\If{$N$ is not even}{
		\Return error: $N$ should be a power of two
	}
	$q = \verb|FFT(all even terms in input list)|$\\
	$r = \verb|FFT(all odd terms in input list)|$\\
	$Y = \verb|the empty list|$\\
	\For{$k = 0$; $k < N/2$}{
		$z = e^{-2k\pi i / N} = \cos(\frac{-2k\pi}{N}) + i \sin(\frac{-2k\pi}{N})$\\
		$Y_k = q_k + zr_k$\\
		$Y_{k + N/2} = q_k - zr_k$\\
		$k \leftarrow k + 1$\\
	}
	\Return Y
	\caption{The Fast Fourier Transform (FFT) algorithm
\label{alg:fft}}
\end{algorithm}

In order to analyse the complexity of the FFT algorithm, we will apply the \emph{master theorem} - as shown in the lectures of this module. The motivation to do so is that the algorithm is recursive, and may be expressed as a recurrence relation. First, we must show what that recurrence relation is, and then figure out which case of the master theorem it lies in.
\\\\
The time taken for the FFT algorithm is proportional to the following recurrence relation:
$$T(n) = 2T(\frac{n}{2}) + \Theta(n)$$
where $T(n)$ is the time taken for an input list of size $n$. The reasoning behind this is clear from the pseudocode in algorithm \ref{alg:fft}. The first term of the recurrence relation, $2T(\frac{n}{2})$ arises because, as algorithm \ref{alg:fft} shows, two recursions of itself are computed on an input which is \emph{half} the size of the original. I.e. $\verb|FFT|$ is computed for all items in the input with an even index, and the same again for all items with an odd index.

After the recursion, computation is done to find multiple \emph{roots of unity}. Similarly to the case of algorithm \ref{alg:dft}, we shall say that this takes a constant amount of time to evaluate, as it runs a fixed set of operations on an item of data with a fixed size. This computation happens precisely $N/2$ times, except in the very specific case that $N = 1$ which is constant. However this special case still applies to the lower bound of $\Omega(n)$ because it is only the case when $n = 1$ and $\Omega(n) = \Omega(1)$ anyway. Hence, the second term - the computational overhead after that of recursion - is of the order $\Omega(n)$, and $O(n) \Rightarrow \Theta(n)$; bringing us to the recurrence relation shown.
\\\\
Given this recurrence relation, we must next find out which case of the master theorem (one, two, or three) applies.
\paragraph{Case 1}
For a recurrence relation of the form...
$$T(n) = aT(\frac{n}{b}) + f(n)$$
...the following statement must also apply in case one:
\[
f(n) = O(n^{\log_b{a - \epsilon}}) \text{ where } \epsilon > 0
\]
We know the values of $a$ and $b$ to both be $2$, it is easy to see. But, the second term, $\Theta(n)$ does not apply to case one because $\Theta(n) = O(n^{\log_2{2 - \epsilon}})$ is only the case when $\epsilon = 0$.

\paragraph{Case 2}
Next, the following statement must apply in case two:
\[
f(n) = O(n^{\log_b{a}}\log^k{n}) \text{ for some } k \geq 0
\]
Using the known values of $a$ and $b$, and selecting a value of $k = 0$, we get:
\[
O(n(\log{n})^0) = O(n)
\]
showing that our second term, $\Theta(n)$ satisfies case two.

\paragraph{Conclusion}
For case two of the master theorem, the complexity is asymptotically tightly bound as $\Theta(n^{log_b{a}}\log^{k + 1}n)$. Substituting the values from the above discussion, $a = 2$, $b = 2$, $k = 0$ we find that the complexity for FFT is $$\Theta(n\log{n})$$

\subsection{Average Case Analysis}\label{sec:fft-average}
\begin{figure}[!htbp]
	\centering
	\begin{minipage}[b]{0.4\textwidth}
		\begin{tikzpicture}
		\begin{axis}[ xmin=0, ymin=0, xlabel=Input Length, ylabel=Time ($\mu$s), legend style={at={(1,-0.1)},anchor=north} ]
		\addplot table [ignore chars=\%, x=Param: inputLength, y=Score, col sep=comma] {csv/pc1-result-fft.csv};
		\addlegendentry{PC1}
		\end{axis}
		\end{tikzpicture}
		\caption{FFT - PC1\label{fig:compar3}}
	\end{minipage}
	\hfill
	\begin{minipage}[b]{0.4\textwidth}
		\begin{tikzpicture}
		\begin{axis}[ xmin=0, ymin=0, xlabel=Input Length, ylabel=Time ($\mu$s), legend style={at={(1,-0.1)},anchor=north} ]
		\addplot table [ignore chars=\%, x=Param: inputLength, y=Score, col sep=comma] {csv/pc2-result-fft.csv};
		\addlegendentry{PC2}
		\end{axis}
		\end{tikzpicture}
		\caption{FFT - PC2\label{fig:compar4}}
	\end{minipage}
\end{figure}

\section{Empirical Observation for the DFT/FFT Threshold}\label{sec:threshold}
For the sake of comparison, the two average cases described in sections \ref{sec:dft-average} and \ref{sec:fft-average} are shown here in the same plot.

\begin{figure}
	\begin{center}
		\begin{tikzpicture}
		\begin{axis}[ xmin=0, ymin=0, xlabel=Input Length, ylabel=Time ($\mu$s), legend style={at={(1,-0.1)},anchor=north} ]
		\addplot table [ignore chars=\%, x=Param: inputLength, y=Score, col sep=comma] {csv/pc1-result-dft.csv};
		\addlegendentry{PC1 DFT}
		\addplot table [ignore chars=\%, x=Param: inputLength, y=Score, col sep=comma] {csv/pc1-result-fft.csv};
		\addlegendentry{PC1 FFT}
		\addplot table [ignore chars=\%, x=Param: inputLength, y=Score, col sep=comma] {csv/pc2-result-fft.csv};
		\addlegendentry{PC2 FFT}
		\addplot table [ignore chars=\%, x=Param: inputLength, y=Score, col sep=comma] {csv/pc2-result-dft.csv};
		\addlegendentry{PC2 DFT}
		\end{axis}
		\end{tikzpicture}
		\caption{DFT and FFT for PC1 and PC2\label{fig:compar5}}
	\end{center}
\end{figure}

\begin{figure}[!htbp]
	\centering
	\begin{minipage}[b]{0.4\textwidth}
		\begin{tikzpicture}
		\begin{axis}[ xmin=0, ymin=0, xlabel=Input Length, ylabel=Time ($\mu$s), legend style={at={(1,-0.1)},anchor=north} ]
		\addplot table [ignore chars=\%, x=Param: inputLength, y=Score, col sep=comma] {csv/pc1-result-dft.csv};
		\addlegendentry{PC1}
		\addplot table [ignore chars=\%, x=Param: inputLength, y=Score, col sep=comma] {csv/pc2-result-dft.csv};
		\addlegendentry{PC2}
		\end{axis}
		\end{tikzpicture}
		\caption{Naive DFT for PC1 and PC2\label{fig:compar6}}
	\end{minipage}
	\hfill
	\begin{minipage}[b]{0.4\textwidth}
		\begin{tikzpicture}
		\begin{axis}[ xmin=0, ymin=0, xlabel=Input Length, ylabel=Time ($\mu$s), legend style={at={(1,-0.1)},anchor=north} ]
		\addplot table [ignore chars=\%, x=Param: inputLength, y=Score, col sep=comma] {csv/pc1-result-fft.csv};
		\addlegendentry{PC1}
		\addplot table [ignore chars=\%, x=Param: inputLength, y=Score, col sep=comma] {csv/pc2-result-fft.csv};
		\addlegendentry{PC2}
		\end{axis}
		\end{tikzpicture}
		\caption{FFT for PC1 and PC2\label{fig:compar7}}
	\end{minipage}
\end{figure}

\begin{figure}[!htbp]
	\centering
	\begin{minipage}[b]{0.4\textwidth}
		\begin{tikzpicture}
		\begin{axis}[ xmin=0, ymin=0, xlabel=Input Length, ylabel=Time ($\mu$s), legend style={at={(1,-0.1)},anchor=north} ]
		\addplot table [ignore chars=\%, x=Param: inputLength, y=score per input length, col sep=comma] {csv/pc1-result-fft.csv};
		\addlegendentry{PC1}
		\addplot table [ignore chars=\%, x=Param: inputLength, y=score per input length, col sep=comma] {csv/pc2-result-fft.csv};
		\addlegendentry{PC2}
		\end{axis}
		\end{tikzpicture}
		\caption{$\frac{T}{N}$ against input length $N$\label{fig:compar6}}
	\end{minipage}
	\hfill
	\begin{minipage}[b]{0.4\textwidth}
		\begin{tikzpicture}
		\begin{axis}[ xmin=0, ymin=0, xlabel=Input Length, ylabel=Time ($\mu$s), legend style={at={(1,-0.1)},anchor=north} ]
		\addplot table [ignore chars=\%, x=Param: inputLength, y=score per input squared, col sep=comma] {csv/pc1-result-dft.csv};
		\addlegendentry{PC1}
		\addplot table [ignore chars=\%, x=Param: inputLength, y=score per input squared, col sep=comma] {csv/pc2-result-dft.csv};
		\addlegendentry{PC2}
		\end{axis}
		\end{tikzpicture}
		\caption{$\frac{T}{N^2}$ against input length $N$\label{fig:compar7}}
	\end{minipage}
\end{figure}

\begin{figure}[h]
	\begin{center}
		\begin{tabular}{ | c | c | c | }
			\hline
			specification & PC1 Value & PC2 Value \\
			\hline
			CPU & x86\_64 Intel(R) i7-4712HQ @ 2.30GHz & x86\_64 Intel(R) i5-3470S @ 2.90GHz\\
			\hline
			RAM & 16GB = 2 $\times$ 8GB @ 1600MHz DDR3 & TODO add RAM specs \\ 
			\hline
			OS & Ubuntu \emph{xenial} 16.04.2 & Fedora release 24\\
			\hline
			Java Version & 1.8.0\_121 & 1.8.0\_121\\
			\hline
		\end{tabular}
	\end{center}
	\caption{relevant system specifications used in provided analysis\label{fig:specs}}
\end{figure}
\nocite{*}

\bibliography{ref}
\bibliographystyle{apalike}

\end{document}
